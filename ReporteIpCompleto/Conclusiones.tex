\section{Conclusiones}
	A lo largo del desarrollo de esta pr\'actica que fue a lo largo del semestre, pudimos ver la importancia y diferencia de los protocolos orientados a bits y los orientados a bytes, ya que un protocolo orientado a bits es mas eficiente ya que utilizamos y aprovechamos cada uno de los bits de los mensajes sin que haya desperdicio, claro que esto es m\'as dificil sin embargo aprovechas mas los recursos, en cambio uno orientado a bytes es mas sencillo de utilizar, analizar y dise\~nar, sin embargo hay desperdicio de bits, y no uno peque\~no este suele ser muy grande; ejemplos de esto esta el protocolo LLC que es orientado a bits y el protocolo IP que es orientado a bytes.
\vskip 1pt
	Antes de este curso de redes ten\'ia una idea de que era el encapsulamiento y desencapsulamiento, sin embargo no sab\'ia como funcionaba, antes solo decia que se tenian que encapsular los mensajes para que lleguen al remitente y desencapsularlos para que el remitente pueda leer esos datos, sin embargo, no es suficiente, por ello en esta materia vimos como se encapsulaban y transportaban por medio de los protocolos, lo cual me parecio correcto para que seamos capaces de observar de mejor manera el proceso que siguen los protocolos.
\vskip 1pt
	Alguna vez te hab\'ias puesto a pensar que cuando mandas un whatsapp con un \textit{ola k ace} ocurr\'ia todo este proceso entre tu host y el host receptor? Y si despu\'es mandas \textit{?} y despues otro con un emoji \textit{:)}, es decir,  3 diferentes mensajes a cada uno de esos les pasa el encapsulamiento. Si consideraramos de manera general se transmiten por TCP y el encapsulado desde capa fisica hasta transporte, llena la siguiente tabla  y una vez llena, que puedes decir como reflexi\'on?
\vskip 1pt
\begin{table}[h]
	\begin{center}
	\begin{tabular}{| c | c | c | c |}
			Mensaje & Cantidad de caract\'es & Bytes agregados en las cabeceras & Total de Bytes para el Mensaje \\ \hline
			Ola k ace & 9 & 9 & 78 \\
			? & 1 & 1  & 74 \\
			:) & 1 & 1 & 74 \\ \hline
	\end{tabular}
	\caption{Mensaje y cabeceras}
	\label{tab:mensaje}
	\end{center}
\end{table}
\vskip 1pt
	En la siguiente tabla podemos ver que la cantidad de car\'acteres es proporcional al numero de bytes a la cantidad de caracteres, ya que a cada caracter le corresponde un byte gracias a ASCII, en donde primero se convierte a un hexadecimal que es el del c\'odigo que le corresponde y en el total de bytes para el mensaje tenemos los que fueron a\~nadidos en opciones. Por ello somos capaces de observar que es importante esta parte de la encapsulacion.
	\vskip 1pt
		El operador $*$ es un operador aritmetico en c, el cual nos multiplica los valores que le digamos, mientras que el $<$$<$ es un operador a nivel de bit en especifico este hace un corrimiento a la izquierda, sin embargo si nosotros realizamos las operaciones T[12]$*$256 + T[13] y T[12]$<$$<$8 $|$ T[13], tenemos el mismo resultado, esto se debe a que multiplicamos el valor de nuestro unsigned char por 256 que nos moveria 8 bits a la izquierda y le sumamos el valor de T[13], mientras que el operador $<$$<$ directamente nos da el recorrimiento a la izquierda y el operador $|$ le aplica un or logico a nuestra opercion, obvio en mas eficiente en este caso los operadores a nivel de bit q los operadores aritmeticos ya que es menos trabajo para el procesador
	\vskip 1pt
	Si Betito se quiere comunicar con Alicia a traves de la red mas grande del mundo, que es la red de redes, se clasificaria como WAN, seria de tipo mixto, y seria por fibra optica, para comunicarse a grandes distancias, pero no es tan sencillo ya que es una gran red compuesta de redes mas peque\~nas por ejemplo podriamos decir que estan en diferente subred y necesitan de un router con el DG para comunicarse, simplemente las variables son demasiadas para poder llegar a una conclusion definitiva asi que concluimos que sin mas informacion no podemos decir exactamente todo el proceso que va a hacer.